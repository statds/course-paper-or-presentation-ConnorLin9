\documentclass[12pt, letterpaper]{article}
\usepackage{graphicx}
\usepackage{amsmath}
\usepackage{natbib}
\usepackage{url}

\begin{document}

\title{Analysis of Airbnb Pricing Factors in the United States}
\author{Thomas Lin\\
Department of Statistics, University of Connecticut}
\maketitle

\section*{Introduction}
With a pervasive presence across the United States and boasting 7 million listings, Airbnb emerges as a transformative force in the nation's sharing economy. Offering a diverse array of accommodations, from urban apartments to rural retreats, Airbnb encapsulates the essence of curated travel experiences within the U.S. For hosts, pricing becomes a delicate equilibrium, intricately shaping the competitiveness and profitability of their listings. Confronted with many choices, travelers navigate pricing considerations that intertwine with location and amenities.

Beyond individual transactions, Airbnb's pricing dynamics reverberate through the fabric of urban planning and community dynamics in the United States, presenting a nuanced challenge for policymakers. Our research delves into the intricate tapestry of Airbnb pricing within the U.S., systematically unraveling the determinants influencing this dynamic marketplace. The overarching goal is to contribute significantly to the academic understanding of the sharing economy within the United States and provide practical insights for hosts, travelers, and policymakers navigating the ever-evolving landscape of travel and hospitality within the country.

\section*{Background}
The exponential growth of Airbnb, with over 7 million listings across 220 countries, has reshaped the global travel and lodging sector. The platform's ubiquity underscores its transformative impact, offering diverse accommodations from modest apartments to luxurious villas. Within this dynamic marketplace, pricing strategies exhibit considerable variability. Hosts, acting as independent entrepreneurs, employ various approaches influenced by factors such as location, property type, and seasonal demand. Location plays a pivotal role, with properties in different regions commanding distinct pricing structures.

Property type further shapes Airbnb pricing, reflecting the diverse range of accommodation styles available on the platform. Seasonal demand introduces complexity, with travel patterns influencing booking volume and optimal pricing strategies for hosts throughout the year. Existing studies highlight the impact of property attributes, location characteristics, and host behavior on pricing. However, a comprehensive, data-driven exploration of Airbnb pricing dynamics in the United States is needed. This research project aims to fill this gap by delving into pricing trends and uncovering patterns within the Airbnb dataset. The goal is to provide actionable insights for hosts, travelers, and policymakers navigating the dynamic realm of short-term accommodations.

\section*{Specific Aims}
This research project is designed to achieve the following specific aims for the United States:

\begin{enumerate}
  \item \textbf{Data Collection}: Gather and curate a comprehensive dataset of Airbnb listings in the United States, incorporating detailed information on property attributes, location characteristics, host profiles, and pricing details.
  \item \textbf{Data Analysis}: Apply advanced analytical techniques to identify primary factors influencing Airbnb pricing in the United States. Investigate relationships between pricing and property type, location, occupancy rates, and seasonality variables.
  \item \textbf{Predictive Models}: Develop robust predictive pricing models specific to the United States to estimate Airbnb rental prices based on identified determinants. Empower hosts and travelers with tools to make informed decisions within the U.S. market.
\end{enumerate}

\section*{Data Collection}
To ensure the integrity and relevance of the dataset, this study will focus on collecting data from Airbnb listings in the United States. In addition, publicly available data from the Airbnb website will be utilized. The comprehensive dataset will include property attributes, location characteristics, host profiles, pricing details, historical occupancy rates, and booking data.

\section*{Methods}
The analysis will encompass a diverse set of statistical and machine-learning methods, ensuring a rigorous examination of Airbnb pricing factors within the United States:

\begin{enumerate}
  \item \textbf{Descriptive Statistics}: Conduct a comprehensive descriptive analysis of the dataset specific to the United States, summarizing variables, investigating central tendencies, spread, and distribution, and identifying potential outliers.
  \item \textbf{Regression Analysis}: Utilize multiple regression techniques to model the relationship between pricing and independent variables within the United States. Identify significant factors influencing Airbnb pricing in the U.S.
  \item \textbf{Machine Learning Algorithms}: Implement advanced machine learning algorithms to build predictive pricing models specific to the United States. These models will provide a granular understanding of how various factors collectively affect Airbnb pricing within the U.S. market.
  \item \textbf{Time-Series Analysis}: Incorporate time-series analysis to account for seasonality and temporal trends, revealing pricing patterns over time within the United States.
\end{enumerate}

\section*{Graphs and Visuals}
RStudio will be employed to create compelling data visualizations specific to the United States, enhancing the presentation of results. Visualizations, including scatter plots, histograms, and time series plots, will serve as valuable tools for conveying complex information in a user-friendly manner within the U.S. context.

\section*{Results}
This first draft is in the early stages of analysis, and actual results are pending. The results section will be populated with findings and insights derived from the comprehensive data analysis outlined in the methodology within the context of the United States.

\section*{Discussion}
Anticipating the results, the discussion will delve into the implications for hosts, travelers, and policymakers within the United States. How can hosts optimize their pricing strategies, and how can travelers benefit from making informed decisions when booking accommodations within the U.S.? Additionally, what are the broader implications for local tourism and housing policies specific to the United States?

\section*{Conclusion}
In conclusion, this research project aims to provide invaluable insights into the determinants of Airbnb pricing within the United States. By understanding how various factors influence rental rates in the U.S. market, hosts can optimize their pricing strategies, travelers can make informed decisions, and policymakers can shape effective regulations for sustainable growth in short-term rental markets.

\section*{Future Work}
This research project opens avenues for future studies within the United States. Future research could delve deeper into state-specific variations in pricing factors or explore the impact of local regulations on Airbnb pricing dynamics in different regions of the country. These potential areas represent exciting opportunities for further exploration and understanding within the U.S. context.

\begin{thebibliography}{}

\bibitem{dogru2020}
Dogru, T., Zhang, Y., Suess, C., Mody, M., Bulut, U., \& Sirakaya-Turk, E. (2020). What caused the rise of Airbnb? An examination of key macroeconomic factors. \textit{Tourism Management}, \textit{81}, 104134. \url{https://doi.org/10.1016/j.tourman.2020.104134}

\bibitem{toader2021}
Toader, V., Negrușa, A. L., Bode, O. R., \& Rus, R. V. (2022). Analysis of price determinants in the case of Airbnb listings. \textit{Economic Research-Ekonomska Istraživanja}, \textit{35}(1), 2493-2509. \url{https://doi.org/10.1080/1331677X.2021.1962380}

\end{thebibliography}

\end{document}
